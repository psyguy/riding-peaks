To estimate the multilevel cosinor model, we make use of a Bayesian estimation approach by running Hamiltonian Monte Carlo sampling (HMC; \cite{betancourt_2018_ConceptualIntroductionHamiltonian}) for $D$ iterations (see Appendix \ref{app:fitting-multilevel-cosinor} for details).
At each iteration $d$ ($d = 1, 2, \dots, D$) and for each individual $i$ ($i = 1, 2, \dots, N$), we obtain random draws for the individual-specific parameters of the linear cosinor model.
Hence, at every $d$, we have for every person a draw for their MESOR (i.e., $M_i^{(d)}$), amplitude (i.e., $A_i^{(d)}$), phase (i.e., $\phi_i^{(d)}$), and logarithm of residual SD (i.e., $\rho_i^{(d)}$).
We visualize these samples in Figure \ref{fig:cube-t-ind} as cells in the three-dimensional matrix that forms an empirical approximation of the joint posterior distribution of the individual parameters.
To obtain an estimate for the level-1, person-specific parameters $M_i$, $A_i$, $\phi_i$, and $\rho_i$, we can aggregate per person and per parameter across the draws as visualized in Figure \ref{fig:cube-t-across-d-ind}, and obtain their individual point estimates (i.e., $\widehat{M_i}$, $\widehat{A_i}$, $\widehat{\phi_i}$, and $\widehat{\rho_i}$) using the mean or median.
To quantify individual uncertainty in parameter estimates, we can calculate the 95\% credible interval (95\%CI) of each individual parameter---which, with a 95\% probability, contains the true value of each individual parameter value---by the corresponding 95\% highest density interval (HDI) of its posterior distribution.
Since their distributions are unimodal, the lower and the upper bounds of the 95\%HDI can be obtained by finding the 2.5\% and 97.5\% quantiles of the posterior distribution of the parameter using its sample quantile function.
\begin{figure}[htbp]
\centering
\begin{subfigure}[t]{0.45\columnwidth}
\centering
\scalebox{0.40}{\input{cube-t-ind}}
\caption{Individual-specific parameters of the nonlinear model at each draw.}
\label{fig:cube-t-ind}
\end{subfigure}
\hspace{0.05\columnwidth}
\begin{subfigure}[t]{0.45\columnwidth}
\centering
\scalebox{0.40}{\input{cube-t-across-d-ind}}
\caption{Obtaining level-1 individual-specific parameter estimates.}
\label{fig:cube-t-across-d-ind}
\end{subfigure}
\caption{Obtaining level-1 parameter estimates of the multilevel cosinor model from the empirical posterior joint distribution of person-specific parameters.
Each cell in the left panel represent a single value of $M_i^{(d)}$, $A_i^{(d)}$, $\phi_i^{(d)}$, or $\rho_i^{(d)}$ of individual $i$ at Hamiltonian Monte Carlo (HMC) draw $d$.
In the right panel, colored, bigger cells contain distributional summary statistics  (mean, median, CI limits, and 95\% highest density region [HDR] coverage) for individual-specific parameters $M_i$, $A_i$, $\phi_i$, and $\rho_i$ pulled across the HMC draws.}
\label{fig:aggregations-level1}
\end{figure}

Although these quantities can be easily calculated for the individual MESOR and residual SD, we face several challenges when obtaining them for the individual amplitude and phase.
The first challenge involves the CI of the amplitude:
Given its definition in Equation \ref{eq:calc-amplitude}, all $A_i^{(d)}$ samples are always non-negative, meaning that its 95\%HDI never crosses zero.
As a result, we cannot use it for hypothesis testing regarding the presence of a cyclic trend in the data (which requires evidence that $A_i > 0$) by means of the 95\%HDI coverage.
The second challenge concerns the estimation of individual phase:
The conventional methods of obtaining the mean, median, and quantile function lead to non-intuitive quantities for point estimates and CIs of individual phase, and their values depend on when the cycles are assumed to start.
To illustrate these challenges---and provide solutions thereof---we make use of the two empirical datasets described in the introduction.

\subsection{Quantifying evidence for \texorpdfstring{$A_i > 0$}{Ai > 0}}

Recall that for an individual's amplitude $A_i$ to be zero, their $C_i$ and $S_i$---which can take on positive and negative values---should be simultaneously zero (see Equation \ref{eq:calc-amplitude}).
As a result, to test the hypothesis of the presence of a cosinor trend in the data of individual $i$ (which requires a nonzero amplitude, i.e., $A_i > 0$), we need to consider the individual's joint posterior distribution of $C_i$ and $S_i$ across all draws $d$ and test the hypothesis against a criterion.
In a Bayesian context, the $\alpha$\%HDI (with $\alpha$ often set to 95) coverage criterion can be generalized to a joint distribution by considering the $\alpha$\% highest density region (HDR) of the multivariate posterior distribution;
an $\alpha$\%HDR of the distribution contains $\alpha$\% of its probability mass.
Assuming $\alpha = 95$, our hypothesis testing criterion for nonzero amplitude $A_i$ would be whether the origin at $[0,0]$ falls within the 95\%HDR of the joint posterior distribution of $C_i^{(d)}$ and $S_i^{(d)}$;
if it is outside this region, we may conclude in favor of a trend in the data.\footnote{With a frequentist approach of fitting the cosinor model, the coverage of the 95\% \textit{confidence ellipse} of the sampling distribution of $C_i$ and $S_i$ is tested.
This ellipse is reconstructed using the point estimates and the covariance matrix of the fitted cosinor model and assuming the joint distribution of the estimated $C_i$ and $S_i$ is a multivariate normal distribution.
See \textcite{bingham_1982_InferentialStatisticalMethods} for more details.}
To obtain the $\alpha$\%HDR of the posterior joint distribution of $C_i$ and $S_i$, we can calculate its empirical probability density over a grid on the posterior samples, find the density at each point in the grid, and construct cumulative probability mass at those points to find the regions that correspond to all $\alpha$ values.
To test our hypothesis, we can check whether the origin falls outside the 95\%HDR.

In Figure \ref{fig:joint-posterior-hdr} we show density plots of the posterior joint distribution $C_i^{(d)}$ and $S_i^{(d)}$ of four individuals:
Persons D1-76 and D1-179 from Dataset 1 (with the origin at 75.0 and 98.9\%HDR, respectively), and persons D2-11 and D2-26 from dataset 2 (with the origin at 83.9 and 98.1\%HDR, respectively). 
In each panel, the boundary of the 95\%HDR is marked by a contour line, and a red line segment connects the origin to the joint median of $C_i^{(d)}$ and $S_i^{(d)}$ and has a length of $\widehat{A_i}$ and is at angle $\widehat{\phi_i}$ from the x-axis.
Taking the 95\%HDR coverage criterion, we find no convincing evidence that the time series of persons D1-76 and D2-11 contain a cosinor trend;
in contrast, for persons D1-179 and D2-26, there is some evidence that their time series are characterized by such a periodic trend.

\begin{FlexFigure}[htbp]
\centering
\includegraphics[width=0.9\textwidth]{figures/joint-posterior-hdr.pdf}
\caption{Heatmaps of $\alpha$\% highest density region (HDR) of the posterior joint distribution of $C_i^{(d)}$ and $S_i^{(d)}$ of four individuals.
The contour line shows the boundary of the 95\%HDRs.
The median of the joint distribution is connected to the origin with a red line, which has a length of $\widehat{A_i}$ and is at angle $\widehat{\phi_i}$ from the x-axis.}
\label{fig:joint-posterior-hdr}
\end{FlexFigure}

\subsubsection*{Empirical findings}

To have an intuition of the presence and strength of cycles within each dataset, in Figure \ref{fig:l1-amp-hist-with-hdr}, we show the distribution of person-specific amplitudes $\widehat{A_i}$ (in the upper panels) as well as scatter plots of HDR level of the individuals' amplitudes based on their point estimates (in the lower panels).
In the scatter plots, we also show the percentage of individuals that had HDR levels in the 0-80, 80-85, 85-90, 90-95, and 95-100 intervals.
If we use the 95\%HR coverage criterion, we obtain evidence for a cosinor trend in the data for only one in six individuals (16.8\%) in Dataset 1;
this suggests that, for most individuals in the sample, the PA time series are not characterized by clear cyclic trends.
% We can also observe that individuals with lower amplitudes, also have lower HDR percentiles, meaning that we cannot reliably compare individuals with regards to the strength of their assumed cyclic trends. (\textcolor{red}{I want to make a point here but I don't know about what exactly!})
Conversely, in Dataset 2, 68 individuals (97.1\%) showed strong evidence for the presence of a cosinor trend;
interestingly, for 66 individuals (94.3\%), the origin of their joint posterior distribution of $C_i^{(d)}$ and $S_i^{(d)}$ was outside the 100\%HDR.
The two individuals (persons D2-11 and D2-68) whose origin was outside the 95\%HDR of the joint posterior distribution had almost no measurements around early morning, which was the time that the majority of individuals experience their lowest heart rate.

These results---which could not have been achieved based on the point estimates and 95\%CI coverage of individual amplitudes---show that affective and physiological time series differ with regard to cyclic trends within them and potential usefulness of multilevel modeling of them using the multilevel cosinor model:
Since the majority of momentary PA time series did not show evidence of the presence of such cycles, the obtained level-2 parameter estimates in Dataset 1 may merely reflect the model assumptions rather than a meaningful aggregation of individual-level dynamics, undermining their utility in forming substantive understanding;
on the other hand, given the strong evidence for the presence of cyclic trends in heart rate time series of Dataset 2, we can utilize the parameter estimates of the multilevel model to understand the processes leading to them and test hypotheses based on them.

\begin{figure*}[htbp]
\centering
\includegraphics[width=\textwidth]{figures/l1-amp-hist-with-hdr.pdf}
\caption{Individual-specific amplitudes and the corresponding highest density region (HDR) levels.
The upper panels show the distribution of $\widehat{A_i}$, and the lower panels show the HDR levels of individual-specific amplitude versus their point estimates;
the 80, 85, 90, and 95\%HDR threshold are marked with dashed lines, and the percentage of individuals within the bands is reported.}
\label{fig:l1-amp-hist-with-hdr}
\end{figure*}
